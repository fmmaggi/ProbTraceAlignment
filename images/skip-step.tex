
\documentclass[tikz]{standalone}

\usetikzlibrary{arrows,shapes,automata,petri,positioning}
\usepackage{xcolor}
\definecolor{darkblue}{rgb}{0.2,0.2,0.6}
\definecolor{darkred}{rgb}{0.6,0.1,0.1}
\definecolor{darkgreen}{rgb}{0.2,0.6,0.2}

\definecolor{acsiorange}{RGB}{180,88,26}
\definecolor{acsired}{RGB}{150,0,0}
\definecolor{acsigreen}{RGB}{128,198,54}
\definecolor{acsiblue}{RGB}{43,80,150}

%%%%%%%%%%%%%%%%%%%%%%%%%%%%%%%%%%%%%%%%
\definecolor{swimmyred}{RGB}{206,19,55}
\definecolor{marcofucsia1}{RGB}{203,41,123}
\definecolor{marcofucsia1}{RGB}{203,41,123}
\definecolor{marcofucsia2}{RGB}{153,25,94}
\definecolor{marcofucsia3}{RGB}{102,29,70}
\definecolor{marcoblue1}{RGB}{17,183,225}
\definecolor{marcoblue2}{RGB}{16,163,201}
\definecolor{marcoblue3}{RGB}{5,132,165}
\definecolor{marcoblue4}{RGB}{4,105,131}
\definecolor{marcoblue5}{RGB}{0,77,128}
\definecolor{marcoorange}{RGB}{255,147,0}
\definecolor{marcogreen1}{RGB}{82,174,139}
\definecolor{marcogreen2}{RGB}{56,122,101}


\newcommand{\tcolor}{\color{marcofucsia3}}
\newcommand{\wcolor}{\color{marcofucsia1}}
\newcommand{\pcolor}{\color{violet}}

%%%%%%%%%%%%%%%%%%% PETRI %%%%%%%%%%%%%%%%%%%%%%%%
\usetikzlibrary{arrows,shapes,automata,petri,positioning,calc}

\tikzset{
	place/.style={
		circle,
		very thick,
		draw=acsiblue!90,
		fill=acsiblue!10,
		minimum size=6mm,
	},
    tasktrans/.style={
        transition,
		rectangle,
		very thick,
		fill=marcofucsia3!10,
        draw=marcofucsia3!100,
		minimum width=5mm,
        minimum height=5mm,
		inner ysep=2pt
	},
	transitionH/.style={
		rectangle,
		thick,
		fill=black,
		minimum width=6mm,
		inner ysep=2pt
	},
	transitionV/.style={
		rectangle,
		thick,
		fill=black,
		minimum height=6mm,
		inner xsep=2pt
	},
    tautrans/.style={
      tasktrans,
      fill=black!100,
      draw=black!100,
      font=\color{white},
    },
    arc/.style={
        -angle 90,
        thick,
    },
	state/.style={
		rectangle,
        rounded corners=5pt,
        draw,
		very thick,
		fill=orange!10,
		minimum height=5mm,
		minimum width=10mm,
	},
    link/.style={
        -stealth,
        thick,
    },
	config/.style={
		circle,
        rounded corners=5pt,
        draw,
		very thick,
        fill=black,
		minimum height=2mm,
		minimum width=2mm,
        font=\tiny\color{white}
	},
    hconfig/.style={
		circle,
        rounded corners=5pt,
     	minimum height=2mm,
		minimum width=2mm,
        font=\tiny\color{white}
	},
}



\tikzstyle{joint}=[
 	circle,
	minimum size=1mm,
    draw,
    very thick,
]

\tikzstyle{cjoint}=[
    joint,
    fill=marcoorange!80,
]

\tikzstyle{pjoint}=[
    joint,
    fill=marcoorange!30,
]

\tikzstyle{cstep}=[
    rectangle,
    rounded corners=5pt,
    minimum height=3.5em,
    minimum width=5em,
    very thick,
    draw,
    fill=marcoblue1!40,
    font=\footnotesize
]

\tikzstyle{astep}=[
    rectangle,
    minimum height=4.2em,
    minimum width=5em,
    thick,
    draw,
    fill=marcoblue2!20,
    inner sep=5,
]

\tikzset{database/.style={cylinder,aspect=0.5,draw,rotate=90,path picture={
			\draw (path picture bounding box.160) to[out=180,in=180] (path picture bounding
			box.20);
			\draw (path picture bounding box.200) to[out=180,in=180] (path picture bounding
			box.340);
}}}
\usepackage{bpmn-events}
\usepackage{bpmn-gateways}


\tikzset {
  mytask/.style={
    task,
    draw=marcofucsia3!100,
    fill=marcofucsia3!10,
    font=\scriptsize
  }
}


\begin{document}
	
\begin{tikzpicture}[x=10.5mm,y=15mm,>=stealth',bend angle=45,auto,thick]
	
  \node (start) {};
 
  \node[
    ExclusiveGateway,
    draw,
  ] (loopjoin) at (1,0) {};
    
  \node[
    mytask,
    align=center,
  ] (pi) at (2,0) {
      \begin{tabular}{@{~}c@{}}
        $\tcolor\textsf{pick}$\\
        $\tcolor\textsf{item}$
      \end{tabular}
    };
 
  \node[
    ExclusiveGateway,
    draw,
  ] (aiskip) at (3,0) {};  
  \node[
    above right=-1.5mm of aiskip,
    font=\scriptsize  
  ] {\color{magenta}0.5};
  \node[
    below=2mm of aiskip,
    xshift=-1mm,
    anchor=west,
    font=\scriptsize  
  ] {\color{magenta}0.5};
  
    
  \node[
    mytask,
    align=center,
  ] (ai) at (4,0) {
      \begin{tabular}{@{~}c@{}}
        $\tcolor\textsf{add}$\\
        $\tcolor\textsf{item}$
      \end{tabular}
    };

  \node[
    ExclusiveGateway,
    draw,
  ] (aijoin) at (5,0) {};

  \node[
    ExclusiveGateway,
    draw,
  ] (loopsplit) at (6,0) {};
  \node[
    below right=-2mm of loopsplit,
    font=\scriptsize  
  ] {\color{magenta}0.8};
  \node[
    above=2mm of loopsplit,
    xshift=-1mm,
    anchor=west,
    font=\scriptsize  
  ] {\color{magenta}0.2};

  \node (end) at (7,0) {};

  \draw[sequence] (start) -- (loopjoin);
  \draw[sequence] (loopjoin) -- (pi);
  \draw[sequence] (pi) -- (aiskip);
  \draw[sequence] (aiskip) -- (ai);
  \draw[sequence,rounded corners=5pt] (aiskip) 
      |-
      ++(10mm,-15mm) 
      -|  
      (aijoin);
  \draw[sequence] (ai) -- (aijoin);
  \draw[sequence] (aijoin) -- (loopsplit);
  \draw[sequence] (loopsplit) -- (end);
  \draw[sequence,rounded corners=5pt] (loopsplit) 
      |-
      ++(-10mm,15mm) 
      -|  
      (loopjoin);

  
  \node (startnet) at (7,0) {};
  \node[place] (penter) at (8,0) {};
  \node[
    tasktrans,
    label=below:$\tcolor\textsf{pick item}$,
    label=above:$\wcolor 1.0$,
  ] (t1) at (9,0) {$t_1$};
  \node[place] (ppi) at (10,0) {};
  \node[
    tasktrans,
    label=below:$\tcolor\textsf{add item}$,
    label=above:$\wcolor 0.5$,
  ] (t2) at (11,0) {$t_2$};
  \node[
    tautrans,
    label=below:$\tau$,
    label=above:$\wcolor 0.5$,
  ] (t3) at (11,-1) {$t_3$};
  \node[
    tautrans,
    label=below:$\tau$,
    label=above:$\wcolor 0.2$,
  ] (t5) at (11,1) {$t_5$};
  \node[place] (pai) at (12,0) {};
  \node[
    tautrans,
    label=below:$\tau$,
    label=above:$\wcolor 1.0$,
  ] (t4) at (13,0) {$t_4$};
  \node[place] (pl) at (14,0) {};
  \node[
    tautrans,
    label=below:$\tau$,
    label=above:$\wcolor 0.8$,
  ] (t6) at (15,0) {$t_6$};
  \node (end) at (16,0) {};

  \draw[arc] (startnet) -- (penter);
  \draw[arc] (penter) -- (t1);  
  \draw[arc] (t1) -- (ppi);
  \draw[arc] (ppi) -- (t2);
  \draw[arc] (t2) -- (pai);
  \draw[arc,rounded corners=5pt] (ppi) |- (t3);
  \draw[arc,rounded corners=5pt] (t3) -| (pai);
  \draw[arc] (pai) -- (t4);
  \draw[arc] (t4) -- (pl);
  \draw[arc,rounded corners=5pt] (pl) |- (t5);
  \draw[arc,rounded corners=5pt] (t5) -| (penter);
  \draw[arc] (pl) -- (t6);
  \draw[arc] (t6) -- (end);

\end{tikzpicture}
	
\end{document}