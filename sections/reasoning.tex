\section{Reasoning with Constraint Scenarios}
Constraint scenarios can be used to perform a variety of tasks. We focus here on two fundamental ones: conformance checking and probabilistic constraint entailment.

\subsection{Conformance Checking}
In \declare, the simplest form of conformance checking amounts to check whether a given execution trace satisfies all constraints contained in the model, thus returning a yes/no answer.

In \pdeclare, this notion can be refined by considering the different constraint scenarios and their probabilities. Let $\M = \tup{\tasks,\cset}$ be a \pdeclare model, and $\tau$ be a trace over $\tasks$. The plausible scenarios of $\M$ are pairwise disjoint subsets of the overall set $\tasks^*$ of traces over $\tasks$. Disjointness comes from the fact that every pair of plausible scenarios is so that they disagree about at least one constraint, and no trace can conform with both of them. The complement of the traces accepted by the plausible scenarios then characterizes those traces that are not conforming with $\M$. To assess conformance, we can then proceed as follows:
\begin{inparaenum}[(1)]
\item Check $\tau$ against every plausible scenario of $\M$.
\item If one plausible scenario is so that $\tau$ holds there, output \emph{yes} together with the probability (or range of probabilities) attached to that scenario; the scenario probability gives an indication on whether the trace represents a ``mainstream'' execution of the process, or is instead an outlier behavior.
\item If no such scenario is found, then output \emph{no}.
\end{inparaenum}

\begin{example}
  Consider the \pdeclare model captured by \emph{Case 1} in Table~\ref{tab:cases}. Trace $\tup{\accept,\cancel,\pay}$ does not conform with the model, since paying and canceling are mutually exclusive. Trace $\tup{\accept,\cancel}$ is instead conforming, as it satisfies scenario $1001$. Since this scenario is associated with probability $0.1$, the analyzed trace represents an outlier behavior. Finally, trace $\tup{\accept, \pay, \ship, \ship, \pay, \ship}$ represents a mainstream behavior since it conforms with the most likely scenario $1010$, with probability $0.5$.
\end{example}

\subsection{Constraint Entailment}
It is well-known that \declare and other declarative process modeling languages have the issue of \emph{hidden dependencies} \cite{MPVC11}, namely the fact that constraints may interact with each other in subtle ways. This becomes even more complex in the case of probabilistic constraints. In this light, it becomes crucial to be able to ascertain whether a constraint is implied by a given model. Checking constraint implication in \declare is very simple: this simply amounts to check whether the \LTLf formula of the model implies the given constraint. In the case of \pdeclare, we extend this approach by computing, for a given \LTLf formula, what is the probability with which it is implied by the \pdeclare model. This is done as follows:
\begin{inparaenum}[(1)]
\item Initialize the constraint probability range to $0,0$.
\item For every plausible scenario, check whether the scenario implies the formula of interest in the classical \LTLf sense; if so, update the constraint probability by summing its minimum and maximum to the minimum and maximum probability associated with the scenario.
\item Return the constraint probability range.
\end{inparaenum}

\begin{example}
  Consider again the \pdeclare model captured by \emph{Case 1} in Table~\ref{tab:cases}. We want to check to what extent the model implies that the order is eventually shipped ($\Diamond \ship$). Shipment only occur if a payment occurs before, and therefore this formula is implied only by scenario $1010$, consequently getting a probability of $0.5$.

  We are also interested in checking to what extent the model implies that the order is not rejected ($\neg \Diamond \reject$). This formula holds in all those scenarios where $\constraint{existence}(\reject)$ is false. Therefore, this formula is implied with probability $0.9$.

  Finally, consider the \LTLf constraint $\neg(\Diamond \cancel \land \Diamond \ship)$, expressing the mutual exclusion between \cancel\ and \ship. This constraint is implied with probability $1$, due to the presence of the two crisp constraints $\constraint{not-coexistence}(\cancel,\pay)$ and $\constraint{precedence}(\pay,\ship)$, which must hold in every possible scenario (including the plausible ones).
\end{example}






