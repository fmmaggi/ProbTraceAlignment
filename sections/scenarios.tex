
\section{Reasoning on Time and Probabilities}
As we have seen in the previous section, to reason on conjunctions of probabilistic constraints, i.e., on \pdeclare models, we need to simultaneously take into account the temporal semantics of constraints and their associated probabilities.

Formally, this is done by relying on the probabilistic temporal logic over finite traces \PLTL, recently introduced in \cite{MaMP20}. More specifically, probabilistic constraints as defined here have a direct encoding into the fragment \PLTLz of \PLTL, also investigated in \cite{MaMP20}. We do not delve into the encoding, nor highlight the formal details on how to carry out this combined reasoning. We instead show algorithmically how to accomplish this, noticing that all the algorithmic techniques discussed next are correct thanks to \cite{MaMP20}. Again thanks to \cite{MaMP20}, we also get that, overall, the cost of reasoning on probabilistic constraints has the same complexity of reasoning with standard \LTLf constraints, i.e., \PS in the length of the constraints (this complexity bound is tight).


In the remainder of this section, we fix a \pdeclare model $\model = \tup{\tasks,\cset}$, where $\cset$ is partitioned into \emph{crisp constraints} $\crispc = \set{\tup{\varphi,p} \in \cset \mid p = 1}$ and (genuinely) \emph{probabilistic constraints} $\probc = \set{\tup{\varphi,p} \in \cset \mid p < 1}$. With a slight abuse of terminology, when we use the term ``crisp constraint'', we mean a constraint in $\crispc$, and, when we use the term  ``probabilistic constraint'', we mean a constraint in $\probc$. We also assume that $\crispc$ is a consistent \declare model, i.e., crisp constraints are satisfiable altogether. If not, then $\model$ has to be discarded, as it does not admit any conforming trace.


\subsection{Constraint Scenarios and Consistency of \pdeclare Models}
While crisp constraints must hold in every possible trace, probabilistic constraints may or may not hold (with a ratio specified by their probability). In addition, recall that when a constraint formula does not hold, then its negation must hold. Consequently, in the most general case, $\model$ is a compact description for the $2^{|\probc|}$ standard \declare models, each one obtained by considering all constraint formulae in $\crispc$, and by selecting, for each constraint $\tup{\varphi,p} \in \probc$, whether the constraint formula $\varphi$ or its complement $\neg \varphi$ is assumed to hold.

We call the so-obtained \declare models \emph{(constraint) scenarios}. To pinpoint a specific scenario, we fix an ordering over $\probc$, and we denote the scenario with a binary string of length $|\probc|$, where position number $i \in \set{1,\ldots,|\probc|}$ has value $1$ if the $i$-th probabilistic constraint in $\probc$ must hold, $0$ otherwise.

\begin{example}
  \label{ex:scenarios}
  Consider the \pdeclare model in Example~\ref{ex:prob-interplay}. By fixing the ordering over its probabilistic constraints where $\tup{\constraint{existence}(\accept),0.8}$ is first and $\tup{\constraint{response}(\accept,\pay),0.7}$ is second, we have the following $4$ scenarios:
  \begin{compactenum}
  \item Scenario $00$, where none of the two constraint formulae holds, and is consequently characterized by formula
  $\Box \neg \accept \land \Diamond (\accept \land \Box \neg \pay)$.
    \item Scenario $01$, where the $\constraint{response}$ constraint formula holds while the $\constraint{existence}$ one does not, and so has formula
  $\Box \neg \accept \land \Box (\accept \rightarrow \Diamond \pay)$.
    \item Scenario $10$, where the $\constraint{existence}$ constraint formula holds while the $\constraint{response}$ one does not, and so has formula
  $\Diamond \accept \land \Diamond (\accept \land \Box \neg \pay)$.
     \item Scenario $11$, where both formulae holds ($\Diamond \accept \land \Box (\accept \rightarrow \Diamond \pay)$).
  \end{compactenum}
\end{example}

Among the possible scenarios, only those that are logically consistent, i.e., are associated with a satisfiable formula, have to be retained. In fact, inconsistent scenarios do not admit any conforming trace. Obviously, when checking whether the scenario is consistent, its constraint formulae have to be conjoined with those in  $\crispc$.

\begin{example}
\label{ex:consistent-scenarios}
Consider the $4$ scenarios of Example~\ref{ex:scenarios}. Scenario $00$ has to be discarded because it is logically inconsistent: its formula $\Box \neg \accept \land \Diamond (\accept \land \Diamond \pay)$ is unsatisfiable (it is asking for the presence and absence of $\accept$). The other three scenarios are instead logically consistent.
\end{example}

\begin{example}
\label{ex:consistent-scenarios-2}
Consider the \pdeclare model in Example~\ref{ex:consistent-prob}. Also for this model there are $4$ scenarios, obtained by considering the two $\constraint{existence}$ constraints and their complements. The scenario where both constraints are not satisfied captures those traces where no decision is taken for the order, i.e., the order is not accepted nor rejected. The scenarios where one constraint is satisfied and the other is not account for those traces where a univocal decision is taken for the order. The scenario where both constraints are satisfied, thus requiring acceptance and rejection for the order, is inconsistent, due to the interplay of such constraints and the crisp $\constraint{not-coexistence}$ one. This corresponds to the standard \declare model of Example~\ref{ex:inconsistency}.
\end{example}

We have explicitly used the term \emph{logically} (in)consistent scenarios since there is no guarantee that these scenarios are actually plausible. This depends on their corresponding probabilities, which, in turn, are obtained by suitably combining the probabilities of their constitutive constraints in their positive or complemented form. This is done by enforcing the semantics of constraint probability, which requires to ensure the following: for every probabilistic constraint $\tup{\varphi,p}$, the sum of the probabilities assigned to those scenarios where $\varphi$ must hold must be equal to $p$.

To do so, we construct a system of linear inequalities whose variables represent the probabilities of possible scenarios \cite{MaMP20}. We denote such variables as $x_s$, where $s$ is the boolean string representing the scenario the variable is associated with. By considering a \pdeclare model $\model$, fixing $n = |\probc|$ and writing $i \in \set{0,\ldots,n-1}$ in binary, the system of inequalities $\Lmc_\model$ is:
\begin{align*}
x_i  &\geq  0  && 0\leq i < 2^n
  \tag{\text{$x_i$ are probabilities}} \\
 \sum_{i=0}^{2^n-1}x_i &= 1
  \tag{\text{$x_i$ are probabilities}}\\[2pt]
 \sum_{\text{$j$th position is 1}}x_i &= p_j &&  0\le j <n
  \tag{\text{constraint semantics}}\\
x_i &= 0 &&  \text{if scenario $i$ is logically inconsistent}
\end{align*}
Notably, $\Lmc_\model$ combines, at once, the logical and the probabilistic content of $\model$, on the one hand, imposing that the scenario probabilities agree with the constraint probabilities, and, on the other, forcing logically inconsistent scenario to have probability $0$.

 $\Lmc_\model$ may admit:
\begin{inparaenum}[\it (i)]
\item \emph{no solution}, witnessing that $\M$ is inconsistent;
\item \emph{one solution}, returning the exact probabilities for all the scenarios of $\M$,
\item \emph{multiple (possibly infinitely many) solutions}, witnessing that different probability distributions can be assigned to the scenarios.
\end{inparaenum} To obtain the ranges of probability for each scenario, one can turn the system of inequality into several optimizations problems where each probability variable is minimized and maximized.

It is worth noting that, when $\Lmc_\model$ is solvable, its solutions may force some scenario probabilities to be always equal to $0$. This witnesses the fact that even a logically consistent scenario may not have any conforming trace due to the interplay of constraint probabilities. We call \emph{plausible} those scenarios that have a probability $>0$.

\begin{example}
  Consider again the \pdeclare model in Example~\ref{ex:scenarios} with its 4 scenarios (one of which is logically inconsistent, as discussed in Example~\ref{ex:consistent-scenarios}). The four possible scenarios have corresponding probability variables $x_{00}$,  $x_{01}$, $x_{10}$ and $x_{11}$, constrained by the system of inequalities (we omit the fact that all variables are non-negative):
  $$
  \begin{array}{lclclclcl@{\qquad}l}
  x_{00} &+& x_{01} &+& x_{10} &+& x_{11} &=& 1 &
  \\
         & &        & & x_{10} &+& x_{11} &=& 0.8 &
  \text{semantics of $\tup{\constraint{existence}(\accept),0.8}$}
  \\
         & & x_{01} & &  &+&  x_{11}      &=& 0.7 &
  \text{semantics of $\tup{\constraint{response}(\accept,\pay),0.7}$}
  \\
  x_{00} & &       & &        & &         &=& 0 &
  \text{logical inconsistency of scenario $00$}
  \end{array}
  $$
The system admits a single solution, with $x_{00} = 0$, $x_{01} = 0.2$, $x_{10} = 0.3$ and $x_{11} = 0.5$, the last matching the informal discussion given in Example~\ref{ex:prob-interplay}.
\end{example}

We conclude the section with an informative \pdeclare model example that combines parts of the examples seen so far to capture a non-trivial fragment of an order-to-shipment process. We use parameters for constraint probabilities, then discussing the impact of grounding such probabilities to different actual values.

\begin{table}[t]
\caption{\label{tab:scenarios} Constraint scenarios of the \pdeclare model in Example~\ref{ex:running}, indicating whether they are logically consistent and, if so, providing the (shortest) conforming trace, and the scenario probability.}
{\scriptsize
\begin{tabularx}{\textwidth}{|C@{~~}C@{~~}C@{~~}C|c|l||c|}
\hline
\multicolumn{4}{|c|}{\textsc{scenario}} &
\textsc{logically} &
\multicolumn{1}{c||}{\textsc{shortest conforming}} &
\textsc{scenario}
\\
$\varphi_a$ &
$\varphi_r$ &
$\varphi_{ap}$ &
$\varphi_{ax}$ &
\textsc{consistent} &
\multicolumn{1}{c||}{\textsc{trace}} &
\textsc{probability}
\\
\hline
0 & 0 & 0 & 0 & N & & \\
\hline
0 & 0 & 0 & 1 & N &  & \\
\hline
0 & 0 & 1 & 0 & N &  & \\
\hline
0 & 0 & 1 & 1 & Y & empty trace & $1-p_a-p_r$\\
\hline
0 & 1 & 0 & 0 & N &  & \\
\hline
0 & 1 & 0 & 1 & N &  & \\
\hline
0 & 1 & 1 & 0 & N &  & \\
\hline
0 & 1 & 1 & 1 & Y & $\tup{\reject}$ & $p_r$\\
\hline
1 & 0 & 0 & 0 & Y & $\tup{\accept}$ & $2-p_a-p_{ap}-p_{ax}$\\
\hline
1 & 0 & 0 & 1 & Y & $\tup{\accept,\cancel}$ & $p_a+p_{ax}-1$\\
\hline
1 & 0 & 1 & 0 & Y & $\tup{\accept,\pay,\ship}$ & $p_a+p_{ap}-1$\\
\hline
1 & 0 & 1 & 1 & N &  & \\
\hline
1 & 1 & 0 & 0 & N &  & \\
\hline
1 & 1 & 0 & 1 & N &  & \\
\hline
1 & 1 & 1 & 0 & N &  & \\
\hline
1 & 1 & 1 & 1 & N &  & \\
\hline
\end{tabularx}
}


\end{table}

\begin{example}
  \label{ex:running}
Consider the following order-to-shipment \pdeclare  model:
    \begin{center}
  \resizebox{7cm}{!}{
    \begin{tikzpicture}[node distance=2cm]
      \node[task] (accept) {\accept};
      \node[pwords,above=0mm of accept] {1..*$_{\{p_a\}}$};
      \node[below=0mm of accept,taskfg] {0..1};

      \node[task,left = of accept] (reject) {\reject};
      \node[pwords,above=0mm of reject] {1..*$_{\{p_r\}}$};

      \draw[notcoexistence] (reject) -- (accept);


      \node[task,right=of accept] (pay) {\pay};

      \draw[response,pconstraint]
        ($(accept.east)+(0,2mm)$)
        -- node[above,pwords] {$_{\set{p_{ap}}}$}
        ($(pay.west)+(0,2mm)$);

      \draw[precedence]
        ($(pay.west)-(0,2mm)$)
        --
        ($(accept.east)-(0,2mm)$);



      \node[task,below=1cm of pay] (cancel) {\cancel};

      \draw[precedence,fill=white]
        ($(cancel.west)-(0,1.5mm)$)
        -|
        ($(accept.south east)-(3mm,0)$);

      \draw[response,pconstraint,fill=white]
        ($(accept.south east)-(1mm,0)$)
        |- node[above right,pwords] {$_{\set{p_{ax}}}$}
        ($(cancel.west)+(0,1.5mm)$)
        ;

      \draw[notcoexistence] (pay) -- (cancel);

       \node[task,right=of pay] (ship) {\ship};
       \draw[precedence]
        ($(ship.west)-(0,2mm)$)
        --
        ($(pay.east)-(0,2mm)$);
        \draw[response]
        ($(pay.east)+(0,2mm)$)
        --
        ($(ship.west)+(0,2mm)$);


    \end{tikzpicture}
  }
  \end{center}

To construct the $16$ possible scenarios for this model, the following constraints and \LTLf formulae have to be considered:
\begin{compactitem}[$\bullet$]
\item $\constraint{existence}(\accept)$ with formula $\varphi_a = \Diamond \accept$, and its complement $\Box \neg \accept$;
\item $\constraint{existence}(\reject)$ with formula $\varphi_r = \Diamond \reject$, and its complement $\Box \neg \reject$;
\item  $\constraint{response}(\accept,\pay)$ with formula $\varphi_{ap} = \Box( \accept \rightarrow \Diamond \pay)$,  and its complement $\Diamond(\accept \land \Box \neg \pay)$;
\item $\constraint{response}(\accept,\cancel)$ with formula $\varphi_{ax} = \Box( \accept \rightarrow \Diamond \cancel)$, and its complement $\Diamond(\accept \land \Box \neg \cancel)$.
\end{compactitem}

Table~\ref{tab:scenarios} summarizes the different constraint scenarios, their logical consistency and, in the last column, their probabilities computed by constructing and solving the system of inequalities described above.
Table~\ref{tab:cases} shows instead three different groundings for the constraint probability parameters and their impact on the probabilities of the scenarios. In particular, \emph{Case 1} is so that all the logically consistent scenarios may actually occur, even though with different probabilities. The most likely scenario, accounting for half of the traces, captures the happy path where the order is paid and shipped.
\emph{Case 2} assigns a different probability to $\constraint{response}(\accept,\cancel)$, causing scenario $1000$ to be not plausible anymore, being associated with probability $0$; intuitively, the interplay of constraints and their probabilities makes it impossible to just execute $\accept$ without taking further activities.
Finally, \emph{Case 3} increases the probability of $\constraint{response}(\accept,\cancel)$ even more, resulting in an inconsistent model.





\end{example}

\begin{table}[t]
\caption{\label{tab:cases} Three different groundings for the constraint probabilities used in the \pdeclare model in Example~\ref{ex:running}, and their impact on the scenario probabilities.}
{
\scriptsize
\begin{tabularx}{\textwidth}{|C@{~~}C@{~~}C@{~~}C|C|C|C|}
\hline
\multicolumn{4}{|c|}{\textsc{consistent scenario}} &
\textsc{case1} &
\textsc{case2} &
\textsc{case3}
\\
$\varphi_a$ &
$\varphi_r$ &
$\varphi_{ap}$ &
$\varphi_{ax}$ &
\scriptsize{
  $\begin{array}{@{}r@{~}l@{}}
      p_a =& 0.8\\
      p_r =& 0.1\\
      p_{ap} =& 0.7\\
      p_{ax} =& 0.3
  \end{array}$
  }
&
\scriptsize{
  $\begin{array}{@{}r@{~}l@{}}
      p_a =& 0.8\\
      p_r =& 0.1\\
      p_{ap} =& 0.7\\
      p_{ax} =& 0.5
    \end{array}$
  }
 &
\scriptsize{
  $\begin{array}{@{}r@{~}l@{}}
      p_a =& 0.8\\
      p_r =& 0.1\\
      p_{ap} =& 0.7\\
      p_{ax} =& 0.7
    \end{array}$
  }\\
\hline
0 & 0 & 1 & 1 & 0.1 & 0.1 &  \\
\cline{1-6}
0 & 1 & 1 & 1 & 0.1 & 0.1 & \\
\cline{1-6}
1 & 0 & 0 & 0 & 0.2 & 0 & inconsistent\\
\cline{1-6}
1 & 0 & 0 & 1 & 0.1 & 0.3 & \\
\cline{1-6}
1 & 0 & 1 & 0 & 0.5 & 0.5 &\\
\hline
\end{tabularx}
}
\end{table}





