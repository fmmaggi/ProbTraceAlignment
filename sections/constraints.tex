% !TEX root =  paper.tex

\newcommand{\stoclang}{\rho}
\newcommand{\evlog}{\mathcal{L}}
\newcommand{\nop}{\activity{nop}}
\newcommand{\cancel}{\activity{cancel}}
\newcommand{\pay}{\activity{pay}}
\newcommand{\ship}{\activity{ship}}

\newcommand{\model}{\mathcal{M}}
\newcommand{\cset}{\mathcal{C}}
\newcommand{\crispc}{\cset_{\mathit{crisp}}}
\newcommand{\probc}{\cset_{\mathit{prob}}}

\section{Probabilistic Business Constraints}
\label{sec:probdeclare}

As recalled in Section~\ref{sec:preliminaries}, business constraints captured with \LTLf are interpreted in a crisp way, i.e., they are expected to hold in \emph{every} execution of the process. We now extend constraints with a natural notion of uncertainty introducing probabilistic constraints. Then, we show how this notion can be used to make \declare probabilistic and discuss informally the interplay of multiple probabilistic constraints. %This informal discussion is then mirrored into a formal counterpart in Section~\ref{sec:scenarios}.
%: we quantify what is the probability that a given execution trace of the process will satisfy the constraint. After having introduced the resulting notion

\subsection{Probabilistic Constraints: Definition and Semantics}
For simplicity, we only consider the case of \emph{exact} probability, but all the considerations we do directly carry over the more general case where the probability of a constraint is related to a given quantity with comparison operators ($\leq$, $<$, $=$, and their duals).

\begin{definition}
  A \emph{probabilistic constraint} over a set $\tasks$ of activities is a pair $\tup{\varphi,p}$, where $\varphi$ is an \LTLf formula over $\tasks$ representing the \emph{constraint formula}, and  $p$ is a rational value in $[0,1]$ representing the \emph{constraint probability}.
\end{definition}

Since a probabilistic constraint quantifies \emph{how many} traces should satisfy it, it has to be interpreted over multiple traces that, as a whole, constitute an event log for the process of interest. In particular, the \emph{constraint holds in a log if the ratio of traces in the log that satisfy the constraint formula is equal to the constraint probability}. This naturally leads to interpret the constraint probability statistically as the ratio of conforming vs non-conforming traces contained in a given log.

For simplicity , we stick here with the standard definition of event log, but we could alternatively adopt the stochastic interpretation of an event log, following  \cite{DBLP:conf/bpm/LeemansSA19}.

\begin{definition}
  An \emph{(event) log} over a set $\tasks$ of activities is a multiset of traces over $\tasks$, i.e., a multiset over $\tasks^*$.
\end{definition}
Given a log $\evlog$, we write $\tau^n \in \log$ to indicate that trace $\tau$ appears $n$ times in $\evlog$. Trace $\tau$ belongs to $\evlog$ if $\tau^n \in \log$ with $n > 0$. With these notions at hand, we say that a probabilistic constraint $\tup{\varphi,p}$ holds in a log $\evlog$ or, equivalently, that $\evlog$ satisfies $\tup{\varphi,p}$, if $\sum_{\tau^n \in \evlog, \tau \models \varphi} n = p$.

Note that the probabilistic constraint $\tup{\varphi,p}$ is equivalent to the probabilistic constraint $\tup{\neg \varphi,1-p}$. In fact, given a log $\evlog$, if the ratio of traces in $\evlog$ that satisfies $\varphi$ is $p$, then the remaining $1-p$ traces in $\evlog$ do not satisfy $\varphi$, i.e., they satisfy the constraint complement $\neg \varphi$.

\begin{example}
  Consider the probabilistic constraint $\tup{\constraint{existence}(\accept),0.8}$. It indicates that $80\%$ of the traces in a log contain at least one occurrence of $\accept$ or, equivalently, that $20\%$ of the traces do not contain any execution of $\accept$. This constraint holds in the log:
  $
    \left[
      \begin{array}{l}
        \tup{\accept}^2,
        \tup{\accept,\reject},
        \tup{\reject}^4,
        \tup{\accept,\cancel}^3,
        \tup{\accept,\pay,\ship}^{10}
      \end{array}
    \right]$,~since $16$ out of the $20$ traces contained therein include (at least) one occurrence of $\accept$, i.e., they satisfy $\constraint{existence} = \Diamond \accept$.
\end{example}

\subsection{ProbDeclare and the Issue of Multiple Interacting Constraints}
We now use the notion of probabilistic constraint as the basic building block to lift \declare to its probabilistic version, which we call \pdeclare.

\begin{definition}
A \pdeclare model is a pair $\tup{\Sigma,\mathcal{C}}$, where $\Sigma$ is a set of activities and $\mathcal{C}$ is a set of probabilistic constraints.
\end{definition}

A standard \declare model corresponds to a \pdeclare model where all probabilistic constraints have probability $1$. In the remainder of the paper, when drawing \pdeclare diagrams, we then adopt the following notation:
\begin{inparaenum}[\it (i)]
  \item whenever a constraint has probability $1$, we draw it as a standard \declare constraint;
  \item Whenever a constraint has probability $p<1$, we show it in light blue, and we annotate it with the probability value $p$.
\end{inparaenum}

The main issue that arises when multiple, genuinely probabilistic constraints are present in the same \pdeclare model is that they interact with each other depending on their constraint formulae and probabilities. In particular, to satisfy the probabilistic constraints contained in a \pdeclare model, a log must contain suitable fractions of traces so as to satisfy \emph{all} probabilistic constraints and their probabilities, with the effect that some of these traces may contribute to the computation of the ratios for different constraints.
%
The following examples intuitively illustrate this interplay. The first example shows that inconsistent \declare models may become consistent if the conflicting constraints are associated with suitable probabilities.

\begin{example}
\label{ex:consistent-prob}
Consider the following probabilistic variant of the (inconsistent) \declare diagram shown in Example~\ref{ex:inconsistency}.
\begin{center}
  \resizebox{3.8cm}{!}{
        \begin{tikzpicture}
        \node[task] (accept) {\accept};
        \node[task,right=of accept] (reject) {\reject};
        \node[pwords,left=0mm of accept] {1..*$_{\{0.8\}}$};
        \node[pwords,right=0mm of reject] {1..*$_{\{0.1\}}$};
        \draw[notcoexistence] (accept) -- (reject);
    \end{tikzpicture}
  }
\end{center}
This model contains two mutually exclusive activities, $\accept$ and $\reject$, and indicates that often (in $80\%$ of the cases) $\accept$ is selected, whereas rarely (in $10\%$ of the cases) $\reject$ is selected. This captures a form of probabilistic choice, which also implicitly contemplates that none of the two activities occurs. In fact, from this very simple model, we can infer the following conditions on satisfying logs:
\begin{compactenum}
\item The $\constraint{not-coexistence}$ constraint linking $\accept$ and $\reject$ is crisp, and consequently no trace in the log can contain both $\accept$ and $\reject$.
\item Point 1, combined with the probabilistic $\constraint{existence}$ constraint on $\accept$, means that a trace in the log has $0.8$ probability of containing $\accept$ (which means that $\reject$ will not occur), and $0.2$ probability of not containing $\accept$ (which means that $\reject$ may occur or not).
\item A similar line of reasoning can be applied to the existence of $\reject$, which must appear in $10\%$ of the traces in the log.
\end{compactenum}
All in all, combining all the constraints, we get that the $10\%$ of traces containing $\reject$ must be disjoint from the $80\%$ containing $\accept$. This implicitly means that in the remaining $10\%$ of the traces, none of the two activities occur.
\end{example}

The second example shows that a consistent \pdeclare model may become inconsistent by changing the values of probabilities.

\begin{example}
  \label{ex:inconsistent-prob}
  Consider again the \pdeclare diagram in Example~\ref{ex:consistent-prob}. Clearly, if we change to $1$ the constraint probabilities attached to the two $\constraint{existence}$ constraints, the model becomes identical to that in Example~\ref{ex:inconsistency}, consequently becoming inconsistent. More in general, the model becomes inconsistent whenever the sum of the two probabilities exceeds $1$. This witnesses that there must exist some traces in which both constraints are satisfied, which contradicts the fact that $\accept$ and $\reject$ should not coexist. More precisely, if we denote by $p_a$ and $p_r$ the probabilities attached to the two $\constraint{existence}$  constraints, then there is a probability $p_a+p_r-1$ of having a trace that contains both $\accept$ and $\reject$.  For example, if we set $p_a = 0.8$ and $p_r = 0.3$, we have that $10\%$ of the traces in the log should contain both $\accept$ and $\reject$, which is impossible given the fact that every trace in the log should satisfy $\constraint{not-coexistence}(\accept,\reject)$.
\end{example}

The last example shows that, as customary in models with uncertainty, it is misleading to just consider the probabilities attached to single constraints when one wants to assess the probability of satisfying all of them at once.

\begin{example}
  \label{ex:prob-interplay}
  Consider the following \pdeclare model:
  \begin{center}
  \resizebox{3.5cm}{!}{
    \begin{tikzpicture}[node distance=2cm]
      \node[task] (accept) {\accept};
      \node[pwords,left=0mm of accept,yshift=+3mm] {1..*$_{\{0.8\}}$};
      \node[left=0mm of accept,taskfg,yshift=-3mm] {0..1};

      \node[task,right=of accept] (pay) {\pay};
      \draw[response,pconstraint]
        ($(accept.east)+(0,2mm)$)
        -- node[above,pwords] {$_{\set{0.7}}$}
        ($(pay.west)+(0,2mm)$);
      \draw[precedence]
        ($(pay.west)-(0,2mm)$)
        --
        ($(accept.east)-(0,2mm)$);

    \end{tikzpicture}
  }
  \end{center}
  The model indicates that an order can be accepted at most once, and that often (in $80\%$ of the cases) it is actually accepted. In addition, it captures that with probability $0.7$ it is true that, whenever the order is accepted, then it is also consequently paid (multiple payment instalments are possible, by simply repeating the execution of $\pay$). Finally, payments are enabled only if the order has been previously accepted.

  By looking at the diagram, one could wrongly interpret that in $70\%$ of the cases it is true that the order is accepted and then paid. This is wrong because the $\constraint{response}(\accept,\pay)$ constraint can also be (vacuously) satisfied by a trace that does not contain at all occurrences of $\accept$. A natural question is then: what is the actual probability of observing traces that at some point contain $\accept$ and, later on, $\pay$ (possibly with other activity occurrences in between and afterward)? The answer is that this happens in half of the cases. To justify this non-trivial answer, one has to apply combined reasoning by considering the interplay of $\constraint{response}(\accept,\pay)$ and $\constraint{existence}(\accept)$, with their corresponding probabilities. More specifically, $\constraint{response}(\accept,\pay)$ can be satisfied in this model in two different ways:
  \begin{compactenum}
  \item by not executing at all $\accept$;
  \item by executing $\accept$ (which can be done only once, due to the presence of the crisp $\constraint{absence2}(\accept)$ constraint) and, later on, at least once $\pay$.
  \end{compactenum}
  These two situations, which we will call later on \emph{constraint scenarios}, should altogether cover exactly $70\%$ of the traces, as dictated by the constraint probability attached to $\constraint{response}(\accept,\pay)$. The first scenario must have probability $0.2$, because in $80\%$ of the traces $\accept$ must appear, as dictated by the $\constraint{existence}(\accept)$ constraint and its associated probability.
    But then, the second scenario, which is the one we are interested in, has probability $0.7-0.2 = 0.5$ (half of the traces in the log).
\end{example}

In the next section, we make the reasoning carried out in the discussed examples more systematic, showing how logical and probabilistic reasoning have to be combined towards a single, combined declarative framework.
