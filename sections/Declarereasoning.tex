% !TEX root =  paper.tex
\subsection*{Reasoning Services for ProbDeclare}

\PLTL reasoning services can be applied to \pdeclare. Satisfiability is important to 
check the correctness of \pdeclare models, and to improve discovery algorithms, which typically 
focus on learning single constraints, interleaving learning and satisfiability checks to guarantee 
the overall consistency of their combination \cite{DMMM17}.
%
The computation of the most likely continuations of a partial trace is important when monitoring 
\pdeclare models. Monitoring has been extensively studied for \declare and \LTLf. Specifically, \cite{DDGM14} 
monitors evolving traces against \LTLf formulae through four RV-LTL truth values 
\cite{Bauer2010:LTL}: for a trace $t$, an \LTLf formula $\varphi$ is
\begin{inparaenum}[\it (i)]
\item \emph{permanently satisfied} by $t$ if $t$ and all its extensions satisfy $\varphi$;
\item \emph{temporarily satisfied} by $t$ if $t$ satisfies $\varphi$ but there exists an extension of $t$ that violates $\varphi$;
\item \emph{permanently violated} by $t$ if $t$ and all its extensions violate $\varphi$; and
\item \emph{temporarily violated} by $t$ if $t$ violates $\varphi$ but there exists an extension of $t$ that satisfies $\varphi$.
\end{inparaenum}
Monitoring is based on the automata characterisation of $\LTLf$, and in particular its determinisation.

Since it is not clear how to determinise weighted automata, we cannot lift monitoring to 
\PLTL. However, inspired by RV-LTL we define four feedbacks to be produced during monitoring via 
the most likely extensions of a trace. Given a (partial) trace $t$ and a \PLTL formula $\varphi$, 
we calculate the set $T$ of most likely continuations of $t$ according to \pdeclare 
(Section~\ref{sec:entailment}), and return:
\begin{inparaenum}[\it (i)]
\item \emph{likely satisfaction} (\emph{yes}) if $t$ and all traces in $T$ satisfy $\varphi$;
\item \emph{suggestion to stop} (\emph{stop}) if $t$ satisfies $\varphi$, but all traces in $T$ do not;
\item \emph{likely violation} (\emph{viol}) if $t$ and all traces in $T$ violate $\varphi$;
\item \emph{suggestion to continue} (\emph{go}) if $t$ violates $\varphi$, but all traces in $T$ satisfy $\varphi$.
\end{inparaenum}
These outcomes can be decorated with the probability associated to $T$, indicating the
\emph{strength} of the generated outcome.
The outcome has to be recomputed for every new generated event.

\begin{example}
We monitor the formula $\varphi_{pay}=\Diamond \activity{pay\ order}$
on an evolving trace of the \pdeclare model in Figure~\ref{fig:probdeclare}.
  %
Initially, the verdict for $\varphi_{pay}$ is \emph{go}: it is not yet satisfied, but the most likely continuation does, 
indicating to pick an item, close, accept, and finally pay the order.
If an item is picked and the order is closed, the verdict remains \emph{go}, since the most likely 
continuation is unchanged.
When the order is refused, the verdict changes to \emph{viol}: the most likely extension of the current trace 
indicates to not continue with further events.
Suppose that the order is accepted, changing the previous outcome. The verdict goes back to \emph{go}, 
since the payment has not yet been done, but the most likely extension of the trace indicates it as the next 
event.
Finally the order is paid. The verdict becomes \emph{yes}: $\varphi_{pay}$ is in fact satisfied, and will stay 
so independently on future executions.
\end{example}

  