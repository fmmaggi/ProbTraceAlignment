% !TeX root=../main.tex

\section{Introduction}
\label{introduction}


In the existing literature on conformance checking, a common approach is based on trace alignment \cite{DBLP:conf/edoc/AdriansyahDA11}. However, this approach uses crisp process models as reference models. On the other hand, recently, probabilistic conformance checking approaches are gaining momentum \cite{DBLP:conf/bpm/LeemansSA19,DBLP:conf/icpm/PolyvyanyyK19,DBLP:journals/tosem/PolyvyanyySWCM20}, but
the existing approaches are used to check the degree of conformance of an event log with respect to a stochastic process model
instead of finding trace alignments.
In this paper, we provide for the first time a conformance checking approach based on trace alignments using stochastic reference
models, namely a stochastic Workflow net. Conceptually, this requires to handle the two possibly contrasting forces of the cost of the alignment on the one hand and the
likelihood of the model trace with respect to which the alignment is computed. We consider the important tradeoff between both
aspects.

%Balancing between the likelihood of the model trace with respect to which we are computing the alignment and the cost of the alignment (if the cost of the alignment is too high even if the model trace is very likely applying too many changes in the original trace is in turn not very likely).


Since when aligning an event log with a stochastic net distinct model traces have different probabilities, the retrieval of the best model trace maximizing the combined provision of minimum trace alignment cost and maximum model trace probability might not suffice. Therefore, in this paper, we propose trace alignment approaches that return the best $k$ alignments among all the distinct model traces. To do this, we frame the probabilistic trace alignment problem into the the well-known $k$-Nearest Neighbors ($k$NN) problem \cite{Altman} that refers to finding the $k$ nearest data points to a \textit{query} $x$ from a set $\mathcal{X}$ of \textit{data points} via a distance function defined over $\mathcal{X}\cup\{x\}$.

We introduce two ranking strategies. The first one is based on a brute force approach that reuses existing trace aligners such as \cite{DBLP:conf/edoc/AdriansyahDA11,LeoniM17}, where the (optimal) ranking of the top-k alignments is obtained by computing the Levensthein distance between the trace to be aligned and all the model traces and by multiplying each of these distance by the probability of the corresponding model trace. However, even if this approach returns the best trace alignment ranking for a query trace, the alignments must be computed a-new for all the possible traces to be aligned. Therefore, we propose a second strategy that produces an approximate ranking where $x$ and $\mathcal{X}$ are represented as numerical vectors via an embedding $\phi$. {Then, by exploiting ad-hoc data structures,
%such as Vp-Trees \cite{Fu2000}, Kd-Trees \cite{Maneewongvatana99}, and M-Trees \cite{Ciaccia},
we can retrieve the neighborhood of $x$ in $\mathcal{X}$ of size $k$  by pre-ordering (\textit{indexing}) $\mathcal{X}$  via a distance between the numerical vectors obtained using $\phi$. Thus, we do not need to analyze the entire space, but just start the search from the top-$1$ alignment. If the embeddings $\phi$ for $\mathcal{X}$ are independent from the query of choice $x$, this would not require to constantly recompute the numeric vector representation for $\mathcal{X}$.
%	

%%%%% Proposed part as the last part of the introduction:
%\texttt{\color{red}[TODO]}
%\todo{this is too specific for an introduction; in particular, too many details on how the experiments are done.}
We implemented both strategies\footnote{\url{https://github.com/jackbergus/approxProbTraceAlign}} and perform experiments using a real life event log coming from an hospital system to empirically evaluate the properties of our proposed  strategy. Specifically, we (i) evaluate the correlation between the approximate rankings (using different ways for computing the embeddings) with the optimal ranking, and (ii)~compare the computation time for the exact trace alignment approach against the embedding-based approach.


