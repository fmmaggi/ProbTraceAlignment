% !TeX root=../main.tex

\section{Introduction}\label{introduction}


In the existing literature on conformance checking, a common approach is based on trace alignments. However, it uses crisp process 
models as reference models. On the other hand, recently, probabilistic conformance checking approaches are gaining momentum, but 
the existing approaches are used to check the degree of conformance of an event log with respect to a stochastic process model 
instead of finding trace alignments.
In this paper, we provide for the first time a conformance checking approach based on trace alignments using stochastic reference 
models. Conceptually, this requires to handle the two possibly contrasting forces of the cost of the alignment on the one hand and the 
likelihood of the model trace with respect to which the alignment is computed. We consider the important tradeoff between both
aspects. 

%Balancing between the likelihood of the model trace with respect to which we are computing the alignment and the cost of the alignment (if the cost of the alignment is too high even if the model trace is very likely applying too many changes in the original trace is in turn not very likely).




%%%%% Proposed part as the last part of the introduction: 
\texttt{\color{red}[TODO]}
\todo{this is too specific for an introduction; in particular, too many details on how the experiments are done.}
We perform experiments to empirically evaluate the properties of our proposed  strategy. Specifically, we
%\begin{enumerate}
	(i)~{informally assess the degree of the trace alignment approximation induced by the vector kernel $k_{\phi_\mathcal{P}}$ 
	  compared to the exact probabilistic trace alignment while introducing such embedding (\S\ref{subsec:eta});}
	(ii)~assess the proposed kernel's appropriateness (\S\ref{subsec:katk}) by adding noise to all the traces $\tau$ generated by 
	  a USWN\todo{USWN not defined} $\mathcal{U}$, and evaluating the distance between the trace ranking of $\tau$ and 
	  $\tilde{\tau}$ over the traces of $P$ (\S\ref{subsec:apprp}); and 
	(iii)~compare the computation time for the exact trace alignment approach against the embedding-based approach (\S\ref{subsec:efficio}).
\texttt{\color{red}[TODO]}

Implementation\footnote{\url{https://github.com/jackbergus/approxProbTraceAlign}}